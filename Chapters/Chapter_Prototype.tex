\chapter{Prototype}
Chapter \ref{chapter:gep_neat} explored the theoretical foundation of GEP-NEAT, a hybrid algorithm that integrates the expressive capabilities of Gene Expression Programming (GEP) with the innovation-preserving strategies of NEAT. This chapter shifts focus to the prototype implementation of GEP-NEAT, translating the conceptual framework into a functional system.

\parbreak\noindent Section \ref{sec:proto_introduction} introduces the design rationale behind the construction of the algorithm, outlining key architectural decisions, Section \ref{sec:proto_proto} then presents the full implementation details, covering the core components and mechanisms that drive GEP-NEAT. Following the implementation, the prototype is applied to a series of benchmark problems, with results and visual illustrations discussed in Section \ref{sec:proto_results}. This chapter concludes in Section \ref{sec:proto_conclusion}, summarising the outcomes and reflecting on the effectiveness of the prototype.

\section{Introduction}\label{sec:proto_introduction}
With reference to Chapter \ref{chapter:research_methodology}, this study adopts the Design Science Research (DSR) paradigm, which, as previously discussed, is centered on the creation of innovative artifacts that contribute meaningfully to the existing body of scientific knowledge within a specific domain. DSR emphasises both the design and rigorous evaluation of artifacts, ensuring that they are not only novel but also effective in addressing real-world problems.

\parbreak\noindent As noted by \cite{olivier2009information}, one of the early limitations of the DSR paradigm was the absence of a clearly defined set of operational steps or standardised tool. To address this, Olivier proposed a more structured approach, particularly emphasising the role of conceptual modeling. He argued that models can take various forms, such as descriptive, metaphorical, formal, etc., and may be expressed using principles, scientific notation, or visual languages, depending on research context.

\parbreak\noindent In the context of this research, where the artifact is a genetic algorithm-based system, the implementation naturally lends itself to programmatic modeling. Therefore, the design of the GEP-NEAT prototype is supported by the use of UML class diagrams and sequence diagrams. The class diagram provides a clear and structured representation of the system's components, their relationships, and responsibilities, effectively serving as a blueprint for the artifact's construction. The sequence diagram, on the other hand, captures the control flow and interaction logic between components, offering a high-level view of the algorithm's runtime behaviour. Based on these models, the algorithm is implemented using a programming language, translating the conceptual design into a functional prototype. This approach ensures that the artifact remains aligned with its theoretical foundation while being practically executable and testable.

\parbreak\noindent The evaluation of the prototype is conducted through both quantitative and qualitative methods, in line with DSR's emphasis on rigorous validation. Quantitative evaluation involves measuring the algorithm's performance using standard metrics such as accuracy, precision, convergence speed, and efficiency (\cite{gregar2023research}). These metrics provide objective insights into how well the artifact performs across different problem domains. Complementing this, the qualitative evaluation focuses on the artifact's structural and functional qualities, including modularity, scalability, and innovative design features. As \cite{olivier2009information} emphasises, qualitative insights are essential for assessing how well the prototype aligns with its conceptual model and whether it contributes meaningfully to the broader knowledge base.

\parbreak\noindent In this research, the GEP-NEAT prototype is applied to a variety of problem scenarios to assess its generalisability and effectiveness. The results of these experiments are analysed in relation to the research questions posed earlier in the dissertation. While some questions are directly answered through empirical results, others are addressed through interpretive analysis and discussion, providing a comprehensive understanding of the artifact's capabilities and limitations.

\section{GEP-NEAT Prototype}\label{sec:proto_proto}

\section{Results}\label{sec:proto_results}

\section{Conclusion}\label{sec:proto_conclusion}
