\chapter{Research Methodology}\index{chapter:research\_methodology}\label{chapter:research_methodology}
This chapter outlines the methodological framework used to guide the development, implementation, and evaluation of the GEP-NEAT algorithm, a novel hybrid approach that combines Gene Expression Programming (GEP) and NeuroEvolution of Augmenting Topologies (NEAT). To ensure methodological relevance, the research is structured around the Design Science research (DSR) paradigm, which supports the iterative development of innovative artifacts to be used in real-world application. This is complemented by Olivier's insight on DSR, which provides a more refined approach in implementing DSR pragmatically. This chapter is organized in two sections. The first presents the design science methodology and application. The second discusses DSR from Olivier's perspective.

\section{Design Science Research}
Design Science Research (DSR) is a research paradigm centered on the creation of innovative artifacts that contribute meaningfully to the existing body of scientific knowledge within a specific domain. According to \cite{hevner2004design}, DSR integrates the principles of relevance, rigor, and iterative design to produce solutions that are both practically useful and theoretically grounded. This methodology is particularly well-suited to algorithmic research, where the objective is to construct novel solutions and evaluate their effectiveness through cycles of design, implementation, and refinement. In the context of this research, the artifact developed is the GEP-NEAT hybrid algorithm, which aims to address the specific limitations in existing neuroevolutionary approaches by combining the strengths of GEP and NEAT.

\parbreak
\begin{figure}[H] % Use [H] to suppress floating and place the figure/table exactly where it is specified in the text
	\centering % Horizontally center the figure on the page
	\includesvg[width=\textwidth]{Figures/Chapter_Research_Methodology/rm_dsr.svg} % Include the figure image
	\caption{Design Science Research Process Diagram (adapted from \cite{hevner2004design})}
	\label{fig:dsr} % Unique label used for referencing the figure in-text
\end{figure}

\parbreak\noindent The use of design science as the chosen research methodology can be examined as 6 process elements as follows with reference to Figure \ref{fig:dsr}:

\begin{enumerate}
    \item \textbf{Problem identification and motivation}: The first stage's aim is to formulate the research problem and justify the necessity of a solution. Importantly, this can be broken down into smaller problems in order for the solution to better capture the problem's complexity. Providing the problem and motivation to the reader accomplishes two things, that is, the solution to the problem is motivated to be pursued and secondly, the reader has a much better understanding of what the intention behind the conducted design, development of the prototype and its respective results are.
    \item \textbf{Objectives of a solution}: The second stage of this methodology is to create a set of objectives based on the problem definition defined above. These objectives can be either, quantitative, qualitative, or both. Quantitative objectives deal with measurable outcomes that can be expressed numerically whereas qualitative objectives are difficult to quantify and focus on the quality or nature of the solution.
    \item \textbf{Design and development}: This stage deals with the creation of an artifactual solution along with detailing the artifact's functionality and architecture which will be used to create the actual artifact.
    \item \textbf{Demonstration}: This stage aims to show the efficacy of the artifact to solve the problem at hand by means of ideologies such as simulation, case studies or experimentation.
    \item \textbf{Evaluation}: This stage's aim is to measure essentially how well the created artifact supports a solution to the problem which involves the comparison of tried and tested real world results to that of the artifacts. As mentioned in the objective phase, a quantitative and qualitative approach can be taken; the quantitative comparison being based on quantifiable metrics, such as convergence speed, solution quality, etc., and the qualitative comparison being based on solution innovations, adaptability, ease of use, etc.
    \item \textbf{Communication}: The final stage is to effectively communicate the following:
    \begin{itemize}
        \item \textit{Problem and it's importance}: This will essentially be the problem statement detailed along with its justification.
        \item \textit{Artifact}: An overview of the artifact.
        \item \textit{Artifact's utility and novelty}: A background and technical literature that make up the artifact will be provided to the reader.
        \item \textit{Rigor of its design}: The way in which the newly formed algorithm/design will be detailed to the reader with explicit explanation in the intricate design choices with mentions to previous results of algorithms the prototype is based on.
        \item \textit{Effectiveness}: An analysis will be done on the constructed artifact with comparison to other existing designs using quantitative and qualitative metrics in order to showcase its use and efficacy to the research field at large.
    \end{itemize}
\end{enumerate}

\section{Olivier's Insight}
While Design Science Research (DSR) provides a high-level framework for the creation and evaluation of innovative artifact, it lacks a fixed set of operational steps or tools. To address this, \cite{olivier2009information}, have proposed complementary activities that can be integrated into the DSR process to support knowledge construction, problem exploration, and artifact validation. Each activity contributes to a different phase of the research process, from identifying the problem space to validating the proposed solution.

\subsection{Literature Review}
\noindent The research process begins with a comprehensive review of existing literature, which serves as the foundation for identifying gaps in current knowledge and framing the research problem. In line with Olivier's view, the literature view is not a one-time task, but an ongoing process of gathering, filtering, and synthesizing information. It enables the researcher to build a solid theoretical foundation, avoid redundant approaches, and identify opportunities for innovation.

\parbreak\noindent The literature review focuses on three core areas, that is, evolutionary computation, neuroevolution, and gene expression programming. By critically analysing existing work in these domains, the review highlights the limitations of current approaches and motivates the development of a hybrid solution. In selecting sources, priority is given to peer-reviewed journal articles, followed by conference proceedings, textbook, and reputable online resources. While blogs and traditional \textit{'google'} searches are generally treated with caution and used only as a means to support academic findings.

\parbreak\noindent The literature review also serves several strategic purposes, that is, it helps to define the scope of the research problem, identify methodological approaches, avoid unproductive directions, and uncover new lines of inquiry. In the context of DSR, it provides the initial input for the design cycle by informing the development of the conceptual model and guiding the evaluation criteria for the artifact designed.

\subsection{Conceptual Modeling}
Once the research problem is clearly defined, the next step is to develop a conceptual model that captures the essential components of the proposed solution. In this context, a model serves as a structured representation of the system or process under investigation. It helps to clarify boundaries of the solution space, and provide a blueprint for implementation.

\parbreak\noindent \cite{olivier2009information}, emphasises that models can take various forms depending on the research context. They may be descriptive, metaphorical, or formal, and can be developed using principles, scientific notation, or visual languages. This research employs Unified Modeling Language (UML) diagrams to achieve this as it provides a standardised and widely accepted visual language for representing architecture and behaviour (\cite{koc2021uml}). The diagrams used include:
\begin{itemize}
    \item \textbf{Class and component diagrams}: These diagrams represent the structural composition of the algorithm.
    \item \textbf{Activity diagrams}: These diagrams illustrate the flow of control and decision-making processes.
    \item \textbf{Sequence diagrams}: These diagrams show interactions between components during execution.
\end{itemize}

\parbreak\noindent UML diagrams are chosen for their clarity and ability to convey complex system interactions in a digestible format, ensuring that the model is comprehensible to both technical and academic audiences. In line with \cite{olivier2009information} perspective, models in computer research can be developed through various means, including formal specification, metaphorical representation, or practical design. In this research, the model is primarily constructed through design, using system architecture and algorithmic logic to represent the proposed solution. Where appropriate, descriptive metaphors are used to explain abstract concepts, and formal notation is employed to define algorithmic behaviour.

\subsection{Prototype Development}
With the conceptual model in place, the next step is to construct a working prototype that embodies the proposed solution. In DSR, the prototype serves as a tangible instantiation of the model and provides a means of validating the design through various test and benchmarking mechanisms. As noted by \cite{olivier2009information}, a prototype is not merely a demonstration tool but a vehicle for inquiry. This allows the researcher to explore the behaviour of the system, identify limitation, and refine the design based on feedback.

\parbreak\noindent Prototypes are also recognized as essential tools for reducing uncertainty and improving design outcomes. \cite{camburn2017design}, highlight that prototyping enables real-time feedback, supports iterative development, and facilitates the early identification of design flaws. This allows researchers to test algorithmic behaviour under realistic conditions and to make data-driven decisions about further development. Other research indicates that prototypes provide proof by construction, offering concrete evidence that a theoretical model can be realized in practice. They simply serve as a foundation for further experimentation and analysis, particularly in exploratory research where the goal is to uncover new insights or validate emerging paradigms (\cite{nunamaker1990systems}).

\subsection{Experimental Evaluation}
The final activity in the research process is the empirical evaluation of the prototype. \cite{olivier2009information}, emphasises that experiments in computing research can serve multiple purposes, that is, they can be used to test hypotheses, explore parameter spaces, or validate theoretical models.

\parbreak\noindent The evaluation of the prototype is approached through both quantitative and qualitative methods, in line with DSR principles of the artifact being rigorously tested to validate the effectiveness in solving the identified problem. Quantitative evaluation in this research involves measuring the algorithm's performance using standard metrics such as accuracy, precision, convergence speed, and computational efficiency (\cite{gregar2023research}).  Complementing this, qualitative evaluation focuses on the artifacts structural and functional qualities such as modularity, innovation, and scalability. \cite{olivier2009information} notes that qualitative insights are crucial for understanding how well a prototype aligns with its conceptual model and whether it contributes meaningfully to the body of knowledge.

\section{Conclusion}
By integrating literature review, conceptual modeling, prototype development, and experimental into the DSR framework along with Olivier's insight, this research adopts a comprehensive and methodological rigour approach to artifact construction. Each activity contributes to a different phase of the design cycle and supports overarching goal of developing a novel, effective, and theoretically grounded solution to the problem of evolving neural network architectures. The result is a research process that is capable of producing meaningful contributions to both theory and practice.