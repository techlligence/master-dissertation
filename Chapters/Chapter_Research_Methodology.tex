\chapter{Research Methodology}\index{chapter:research\_methodology}
This chapter outlines the methodological framework used to guide the development, implementation, and evaluation of the GEP-NEAT algorithm, a novel hybrid approach that combines Gene Expression Programming (GEP) and NeuroEvolution of Augmenting Topologies (NEAT). To ensure methodological relevance, the research is structured around the Design Science research (DSR) paradigm, which supports the iterative development of innovative artifacts to be used in real-world application. This is complemented by Experimental Design, which provided a structured approach in testing and validating the generated artifact, and Quantitative Analysis which offers objective metrics for performance evaluation. This chapter is organized in three sections. The first presents the design science methodology and application. The second discusses the role of experimental design in structuring the evaluation process. The third outlines quantitative methods that can be used to assess the hybrid algorithm's performance.

\section{Design Science Research}
Design Science Research (DSR) is a research paradigm centered on the creation of innovative artifacts that contribute meaningfully to the existing body of scientific knowledge within a specific domain. According to \cite{hevner2004design}, DSR integrates the principles of relevance, rigor, and iterative design to produce solutions that are both practically useful and theoretically grounded. This methodology is particularly well-suited to algorithmic research, where the objective is to construct novel solutions and evaluate their effectiveness through cycles of design, implementation, and refinement. In the context of this research, the artifact developed is the GEP-NEAT hybrid algorithm, which aims to address the specific limitations in existing neuroevolutionary approaches by combining the strengths of GEP and NEAT. \bigskip

\noindent The use of design science as the chosen research methodology can be examined as 6 process elements as follows with reference to Figure \ref{fig:dsr}:


\begin{figure}[H] % Use [H] to suppress floating and place the figure/table exactly where it is specified in the text
	\centering % Horizontally center the figure on the page
	\includesvg[width=\textwidth]{Images/Chapter_Research_Methodology/rm_dsr.svg} % Include the figure image
	\caption{Design Science Research Process Diagram (adapted from \cite{hevner2004design})}
	\label{fig:dsr} % Unique label used for referencing the figure in-text
\end{figure}

