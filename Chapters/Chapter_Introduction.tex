\chapter{Introduction}\index{chapter:introduction}

Charles Darwin, an English naturalist, geologist, and biologist theorized the evolution by natural selection suggesting that all life on Earth evolved from common ancestors through a process of adaption and survival of the fittest. Put simply, organisms and animals go through generational mutations and adaptations to survive in the environment they are placed in. Through millions of years of evolution, the human brain began to form which consists of thousands of neurons, small building blocks which act as thresholds to fire signal within the brain. Combining these signals allows complex logic and have evolved to be the development of intelligent brain cognitive function. The computer sicence field has taken these principles and represetned them mathematically, that being, in terms of evolution, evolutionary computing has become a popular field whereby evolutionary processes are mimiced in order to solve optimzation tasks, whereas twith regards to the brain, peceptrons have designed as the model of neurons to simulate cognitave ability which has introduced the world of neural networks. \bigskip

\noindent There are two concepts which this paper is based on, namely gene expression programming (GEP) and the Neuroevolution of augmenting toploiges (NEAT). GEP is an evolutionary algorithm whereby expression trees are represented in a meaningful expression string and undergo generations of adaptations in order to solve an optimzation task. NEAT on the other hand is a architectural search mechanism whereby cancdidate solutions representing a neural netorks topology are evolved through generations to create an optimal neural netowrk to solve a particular task. This research paper proposes a new algorithm, GEP-NEAT which combines the aspects of GEP and GEP-NEAT in an attempt to create an algorithm that obtains strengths from both. Importantly, GEP-NEAT introduces new novelty, which is the representation of innovation numbers as sub-tree configurations. In addition to this, this paper seeks to find a meaninufl way to make use of this representation in the hope to create some powerful metric to be used wtihin the algorithm.

\section{Publications Resulting from this Work}\index{sec:publication\_from\_resulting\_work}
A peer-reviewed conference paper derived from this research was published in the proceedings of the \textbf{8th International Conference on Information Science and Systems (ICISS 2025)}. As an established forum in its eight iteration, ICISS maintains rigorous academic standards through its double-blind peer review process, where both author and reviewer identities are concealed to remove bias and ensure impartial evaluation based solely on scholarly merit. The conference brings together leading researchers across ten interdisciplinary tracks spanning artificial intelligence, data science, and information systems. \bigskip

\noindent The accepted paper, which contributes to the Machine Learning and Artificial Intelligence track, presents the algorithm GEP-NEAT with its innovation number novelty, showcasing the ability to solve the XOR and Cart Pole problem effectively. ICISS 2025 facilitated valuable scholarly exchange through keynote presentations by field leaders, technical workshops, and interdisciplinary discussion bridging academic and real-world application. The conference proceedings are to be published into \textbf{Communications in Computer and Information Sicence (Electronic ISSN: 1865-0937 \& Print ISSN: 1865-0929)} as a proceedings book volume and indexed by EI Compendex, Scopus, INSPEC, SCImago and other databases.

\section{Research Questions}
Write information on research questions here

\section{Research Methodology}
Write information on research methodology here